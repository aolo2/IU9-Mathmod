\documentclass[12pt]{article}

\RequirePackage[l2tabu, orthodox]{nag}

\usepackage[utf8]{inputenc}
\usepackage[russian]{babel}
\usepackage{blindtext, csquotes, indentfirst, amsmath, pgfplots, framed, mathtools}

\title{Лекции по математическому моделированию}
\author{А.Б. Домрачева}
\date{Весенний семестр 2019 г.}

\begin{document}
\maketitle
\thispagestyle{empty}

\newpage
\tableofcontents
\thispagestyle{empty}

\newpage
% . Математическое моделирование и решение инженерных задач с применением ЭВМ
\section{Лекция 1}
\subsection{Характеристики инженерных задач}
Прикладные задачи классифицируют как экономические, инженерные и научные. Отличительные особенности инженерных задач:

\begin{enumerate}
    \item инженерные задачи имеют ярко-выраженную практическую направленность (создание новых конструкций, разработка техпроцесса, минимизация затрат на производство изделий и т.д.);
    \item для инженерных задач характерна необходимость доведения результатов до конкретных чисел, на основании которых можно принимать решения;
    \item для инженерных задач характерен значительный объем выполняемой вычислительной работы, при этом набор методов и их реализаций конечен;
    \item для инженерных задач характерна необходимость создания достаточно сложных математических моделей с использованием современных вычислительных методов. Экономические задачи используют простые типовые модели. Научные задачи, наоборот, выходят за рамки сложных, но типовых моделей и требуют разработки или модификации математического аппарата для их решения и реализации;
    \item инженерные задачи решаются, как правило, специалистами в предметном области (не программистами, не мат-моделистами, не прикладными математиками).
\end{enumerate}

\begin{quotation}
    \enquote{Нельзя научить делать открытия, но можно подготовить открытие}
    \rightline{{--- А.Б. Домрачева}}
\end{quotation}

За это отвечает теория инженерного эксперимента --- один из разделов математического моделирования. В свою очередь, теория инженерного эксперимента делится на теории планирования эксперимента и проведения эксперимента и обработку результатов эксперимента.

\subsection{Проблемы, возникающие при решении инженерных задач}
Проблема первая. Характеристики объектов испытаний, которые требуется определить в результате эксперимента, часто оказываются недоступными непосредственному измерению.

Проблема вторая. Процессы функционирования объекта исследования часто имеют сложный динамический характер и подвержены существенным влияниям изменяющихся условий внешней среды.

Проблема третья. При испытании сложных комплексов необходимо учитывать влияние испытательного, регистрирующего, управляющего оборудования (погрешность измерения).\\

Таким образом, эксперимент может оказаться продолжительным и дорогостоящим. В связи с этим необходим системный подход, предполагающий решение описанных выше проблем с обеспечением необходимой точности решения.\\

\emph{Математическое моделирование} --- метод исследования объектов и процессов реального мира с помощью их приближенных описаний под средством формул (равенств, неравенств, уравнений, логических структур). Такое описание называют математическими моделями.\\

\emph{Математическая модель} --- неотъемлемая часть эксперимента, исследуемая и уже известная техниками (?)  экономической характеристики объекта.\\

Выделяют класс плохо-формализуемых задач, для которого известны собственные методы построения моделей.

Процесс создания математической модели:
\begin{enumerate}
    \item построение математической модели;
    \item постановка исследования и решение вычислительных задач, реализующих модель;
    \item проверка качества моделей на практике, модификация моделей.
\end{enumerate}

\subsection{Построение математической модели ч. 1}
Имея результаты эксперимента, можно установить связь между этими величинами, описываемую на языке математики. При выборе известной математической модели (модель должна быть достаточно полной для изучения свойств исследования или объекта и достаточно простой для реализации на ЭВМ) выбор учитываемых существенных и несущественных факторов носит исследовательский характер: как правило подтверждается характеристиками. В противном случае компромисс матмодели будет определен неверно.

Прежде чем использовать ту или иную известную матмодель необходимо определить тип модели: аналитическая, гипотетическая, имитационная, статическая или динамическая.

\begin{enumerate}
    \item гипотетическая модель --- еще не подтвержденный процесс;
    \item статическая модель --- описывает явления в предположении, что процесс завершен;
    \item динамическая модель --- описывает явления или процессы реального мира при переходе от одного состояния к другому;
    \item имитационная модель --- позволяет имитировать поведение реальной системы на ЦВМ в заданный или формируемый период времени с определенным набором входных данных;
    \item аналитическая модель --- формализация явления или процесса реального мира, учитывающая физико-химические воздействия, протекающие в системе.
\end{enumerate}

\textbf{Пример: баллистическая задача.} Пусть тело (шар) брошено под углом $\alpha$ к поверхности Земли со скоростью $v_0$. Пренебрегая сопротивлением воздуха, считая Землю плоской, а $g = \mathrm{const}$, определить дальность полета.\\

Модель Галилея.

\begin{center}
    картинка
\end{center}

\[\begin{cases}
    x = v_0 \cdot \cos \alpha \cdot t,\\
    y = v_0 \cdot \sin \alpha \cdot t - g \frac{t^2}{2}.    
\end{cases}\]

\[\begin{cases}
    x_1 = 0,\\
    y_1 = 0
\end{cases},
\begin{cases}
    x_2 = \tg \alpha \frac{2 v_o \cos^2 \alpha}{g},\\
    y_2 = \frac{-g}{2 v_0^2 \cos^2 \alpha} x^2 + (\tg \alpha) x
\end{cases}
\]\\

Модель Ньютона: постановка задачи совпадает, но учитывается сила лобового сопротивления воздуха $F = -\beta v^2$, где $\beta = \frac{c \rho S}{2}$ --- коэффициент сопротивления воздуха.

\begin{center}
    картинка
\end{center}

\[
\begin{cases}
    m \frac{du}{dt} = - \beta u \sqrt{u^2 + v^2},\\
    m \frac{dw}{dt} = - \beta w \sqrt{u^2 + v^2} - mg.
\end{cases}
\]

Модель Ньютона более точная в сравнении с моделью Галилея, но не удовлетворяет решению современной баллистической задачи (линейные и угловые параметры, всего шесть неизвестных).

\newpage
\section{Лекция 2}
\subsection{Построение математической модели ч. 2}
Имея результаты эксперимента (выраженные в числовых значениях), необходимо установить связь между этими величинами, описываемую формулам. Модель будут определять величины, которые можно назвать:
\begin{enumerate}
    \item исходными данными;
    \item параметрами модели;
    \item выходными данными.
\end{enumerate}

В моделировании, как правило, модель представляется в виде \enquote{черного ящика}, входные данные --- вектор $\bar{X}$, выходные данные --- вектор $\bar{Y}$, параметры --- $A$.

\begin{center}
    картинка
\end{center}

В соответствии со схемой выделяют типы решаемых задач:

\begin{enumerate}
    \item по входным данным и набору параметров при известном наборе преобразований входных данных в выходные требуется найти решение;
    \item по значениям выходных данных $\bar{Y}$ и фиксированных значениях параметров $A$ (при известном наборе преобразований входных данных в выходные) необходимо оценить набор исходных данных $\bar{X}$;
\end{enumerate}

В модели Галилея решение обратной задачи требует минимум двух переменных в векторе $\bar{Y}$.\\

Ранее как обратные классифицировались и задачи идентификации в узком смысле, то есть такие задачи, в которых известны наборы $\bar{X}$ и $\bar{Y}$, а также тип связи, и достаточно было оценить параметры.

Задача идентификации считается заданной в широком смысле, когда не имеется никаких сведений о типе преобразования (ранее считались не имеющими решения). Пример: в СЛАУ $A\bar{x} = \bar{b}$ найти $A$ при известных $\bar{x}, \bar{b}$.\\

% Как наука моделирование зародилась в 1846 г., когда французкий астроном обнаружил Нептун на основании заонов небесной механики с использованием данных об аномалиях в движении планеты Уран.\\

Даже для простых моделей реализация использует два и более вычислительных алгоритма, то есть после формализации модели на первом этапе необходимо проверить корректность постановки каждой задачи по Адамару-Петровскому и выбрать вычислительный метод решения каждой задачи, обеспечив хорошую обусловленность как самой задачи, так и алгоритма.

После первого расчета по выбранной модели выясняют пригодность для описания процесса или объекта реального мира.\\

Теоретические выводы и результаты, полученные из математической модели, сопоставляются с экспериментами. Может выясниться, что модель адекватна, то есть вполне точно описывает объект исследования, иначе же требуются модификации. На этом этапе решение принимают специалисты в предметной области, оценивая полученную точность и рентабельность модификаций модели.\\

Для проведения эксперимента:
\begin{enumerate}
    \item составляется план;
    \item создается экспериментальная установка;
    \item выполняются контрольные эксперименты;
    \item проводятся серийные опыты;
    \item обрабатываются полученные экспериментальные данные и их результаты.
\end{enumerate}

Натурные эксперименты как правило дороги, часто продолжительны. Дешевой и быстродейственной альтернативой натурному эксперименту является \emph{вычислительный эксперимент}, в основе которого лежит реализация математическом модели на ЦВМ. Возникает коллизия: для построения модели  необходимо провести натурный эксперимент (что бывает невозможно). С другой стороны матмодель строится как альтернатива натурному эксперименту. Эта коллизия считалась нерушимой до середины двадцатого века.

Компромиссом является так называемый \emph{полунатурный эксперимент}, где в качестве модели используются физическая модель объекта, доступные характеристики оцениваются экспериментально (н: число Маха в аэродинамической трубе), другие --- программно.

\textbf{Преимущества} вычислительного эксперимента:
\begin{enumerate}
    \item дешевле;
    \item безопасен;
    \item можно повторить;
    \item предполагает моделирование условий, которые нельзя создать в лаборатории.
\end{enumerate}

\textbf{Ограничения}:
\begin{enumerate}
    \item применимость результата ограничена рамками используемой математической модели.
\end{enumerate}


Создание нового экспериментального цикла предусматривает анализ альтернатив и вариантов эксперимента, а также оптимизацию по заданному критерию или ряду параметров. Для получения оптимального результата может понадобиться большое количество вариантов оптимизации, что делает заведомо продолжительными и вычислительный эксперимент. Для сокращения времени эксперимента строят план вычислительного эксперимента, для чего в настоящее время существует ПО, позволяющее автоматизировать процесс планирования и проведения эксперимента (т.н. проблемно-ориентированное ПО), что вместе с вычислительной техникой и обслуживающим персоналом составляет проблемно-ориентированную информационную систему исследований.

\subsection{Основные этапы решения задачи с применением ЦВМ. Вычислительный эксперимент}
Решение инженерной задачи разделяется на 10 этапов:
\begin{enumerate}
    \item постановка задачи;
    \item построение или модификация математической модели;
    \item постановка вычислительной задачи;
    \item предварительный предмашинный анализ свойств вычислительной задачи;
    \item выбор и построение математической модели;
    \item алгоритмизация и программирование;
    \item отладка программы;
    \item счет по программе;
    \item обработка и интерпретация результатов;
    \item использование результатов (в т.ч. для коррекции матмодели).
\end{enumerate}

\subsection{Постановка задачи}
Имеется общая предварительная формулировка. Установка конкретизируется с уточнением цели исследования. Определяется система координат, в которой задаются входные и выходные данные, начальные и конечные условия. Находится компромисс между полезностью результатов и сложностью достижения цели. Задача формализуется на языке предметной области (но с учетом возможностей современной вычислительной техники).

\subsection{Построение математической модели}
Анализ объекта исследования. Меняется ли он динамически в заданный или формируемый период времени, носят ли переменные (входные или выходные) или алгоритм их преобразования случайный характер, либо нет. На основе анализа цели исследования оценивается необходимость решения прямой, обратной или идентификационной задачи.

\begin{enumerate}
    \item анализируется возможность измерения известных согласно постановке задачи данных; строится схема объекта исследования в виде черного ящика с учетом количества компонент и изменения поведения системы во времени;
    \item выбирается тип моделируемой системы (простая, структурно-сложная, сложная с изменяющимся поведениям, гибридная, \ldots). На схеме указываются все измеряемые переменные и параметры, указывается способ их получения;
    \item определяется тип базовой вычислительной задачи (прямая, обратная, идентификационная);
    \item указывается классифицирующие признаки модели (статическая или динамическая, аналитическая или имитационная, вероятностная или статистическая, \ldots);
    \item по результатам классификации с учетом опыта предметной области выбирается, модифицируется или строится матмодель.
\end{enumerate}

В случае необходимости построения матмодели учитывается опыт смежных областей, а также поиск аналогов.

При построении модели разного типа имеются принципиальные различия, в связи с чем используются типовые методики, в том числе \emph{методика динамической модели} действий или процесса.\\

Методика динамической модели:
\begin{enumerate}
    \item рисуется графическая схема, иллюстрирующая процесс. Например: в выбранной системе координат указываются время начала и, по возможности, время конца измерений. Альтернативой является блок-схема процесса;
    \item по схеме уточняются и обозначаются начальные и конечные входные данные, выходные данные полученные в процессе моделирования, выделяются переменные, которые являются выходными на одном этапе решения задачи и выходными на другом;
    \item определяются физико-химические воздействия на объект, влияющие на изменения выходных данных;
    \item оценивается возможность упрощения модели, ряд воздействий не учитывается;
    \item формализуется общее уравнение модели;
    \item конкретизируется описание каждого отдельного воздействия.
\end{enumerate}

Любая модель предполагает формулировку ограничений, накладываемых на неё. Эти ограничения указываются на этапе постановки задачи и на графической схеме могут обозначаться как вычеркивания физико-химических воздействий на систему.

\newpage
\section{Лекция 3}
На этом этапе уточняется цель исследования и на основе цели формализуются требования к точности вычислений результата и к точности представлений исходных данных. В ряде случаев исходные данные также являются результатом расчетов (если не заданы, не могут быть измерены на основе натурного эксперимента, не выражаются через другие известные либо измеряемые характеристики аналитически). Натурный эксперимент чаще всего является дорогостоящим, объект исследования в нем --- непосредственен. Объект исследования задачи, в ряде случаев --- продолжительный. При этом построение математической модели на основе вычислительного эксперимента приводит к основной коллизии математического моделирования. В этом случае проводят полунатурный эксперимент, объектом исследования которого является физическая модель объекта исследования. Результат, полученный с помощью полунатурного  эксперимента может оказаться недостаточно точным. В связи с чем появляется дополнительная вычислительная задача, итерационно повышающая точность оценки. В ряде случаев уточнения требуют и результат решения всей задачи, то есть в рамках моделирования формализуется основная вычислительная задача, и ряд вспомогательных, определяющих входные (выходные) данные и параметры модели. Проводится классификация задачи (прямая, обратная, идентификационная), в ряде случаев основная задача также разбивается на несколько вычислений, которые решить проще.

\begin{center}
    картинка
\end{center}

Каждая полученная задача анализируется на корректность постановки по Адамару-Петровскому. Если какая-либо из задач является некорректно поставленной, то ее необходимо переформулировать с целью обеспечения существования и единственности решения, оценить обусловленность задачи, анализировать возможность обеспечения устойчивости и хорошей обусловленности решения. Можно переходить к выбору и построению численных методов и последующей алгоритмизации. Перед выбором численного метода задачи пытаются решить количественно, то есть получить количественный результат, но с невысокой точностью. При этом становится известным диапазон значений для каждого решения и вырабатываются ограничения, накладываемые на данные или метод, в том числе подтверждается возможность решения задачи . Чаще решение инженерной задачи сводится к последовательному решению стандарных вычислительных задач для который разработаны эффективные численные методы, но возможны ситуации, когда алгоритм необходимо адаптировать к решению задачи или создать новый метод.

Если имеется несколько методов решения, необходимо сравнить эффективность всей совокупности методов. Для сравнения эффективности выбирают показатели по которым будут сравнивать методы и составляют таблицу эффективности: по горизонтали --- название методов, по вертикали --- значения показателей этого метода.

\subsection{Алгоритмизация и программирование, отладка и счет по программе}

Численный метод содержит гопоно (?) принумн. (?) схему решения, в связи с чем на этом этапе разрабатываются вычислительный алгоритм. Алгоритм проверяется на корректность, обеспечивается устойчивость решения (в том числе вычислительная), результат должен быть достижим за конечное число итераций.

При тестировании алгоритма осуществляется не только поиск синтаксических ошибок, но и проводится валидация результата на тестовых данных. Если часть данных задачи произвольна, то решается задача обратная к поставленной, результат выводится в удобной для восприятия форме. Многократный запуск с тестовыми данными обеспечивает (?) хорошую обусловленность решения и устойчивость. Изменение результата оценивается с помощью допустимой погрешности. В случае оперед. (?) решения вернуться на этап анализа вычислительной задачи. При многократном запуске с измененными данными подтверждается устойчивость решения, проверяются ограничения, наложенные при постановке задачи.

\newpage
\section{Лекция 4}
\subsection{}

На заключительном этапе оценивается соответствие результатов вычислительного эксперимента (в основе лежит построенная матмодель) явлениям или процессам реального мира. Для подтверждения приходится проводить новые натурные испытания, но, в данном случае, коллизия не наблюдается:

\begin{enumerate}
    \item модель построена по результатам другого полунатурного эксперимента;
    \item как правило, выходной натурный эксперимент не так трудоемок, в ряде случаев требуется два-три контрольных значения показателей исследуемого объекта.
\end{enumerate}

Следует отметить, что вычислительный эксперимент с целью подтверждения адекватности модели может проводиться по уникальному плану в отличие от типовой процедуры валидации и верификации результатов расчета на этапе отладки и счета по программе.

\subsection{Статические модели}
%\begin{center}
%картинка (карта высот)
%\end{center}
Статические модели активно применяются при сравнении \enquote{срезов}, полученных в разные моменты времени той или иной динамической модели. Статически приближают: поверхность объекта, а также изменяемые географические (с изменением координат) показатели исследуемых систем. Как правило, статические модели используются в случае сложности динамических, а также для наглядоности представления.

Статическая модель может оказаться лишь иллюстративной частью проекта, использующего физическую модель исследуемого объекта или его динамическую математическую модель.

\subsection{Методика построения статической модели}
\begin{enumerate}
    \item рисуется графическая схема, иллюстрирующая состояние системы в некоторый момент времени в заданной системе координат;
    \item по схеме уточняются входные данные (н: матрица высот), уточняется погрешность представления измеряемой координаты. Выделяются промежуточные переменные, по которым будут оценены выходные, и непосредственно выходные данные.
    
    
    \begin{figure}[ht]
        \centering
        \begin{tikzpicture}
            \selectcolormodel{gray}
            \begin{axis}[]
                \addplot[dashed] coordinates {(0, 5000) (10, 5000)};
                \addplot[dashed] coordinates {(0, 2000) (10, 2000)};
                \addplot[mark=none] coordinates {(0, 500) (10, 500)};
            \end{axis}
        \end{tikzpicture}
        \caption{}
    \end{figure}
    
    В отличие от динамической модели, параметры статической на начальном этапе моделирования не так очевидны, и могут быть оценены по результатам вспомогательных задач.
    \item определяются физико-химические воздействия на объект, переводящие в данный момент систему из предыдущего в текущее состояние;
    \item оценивается возможность формализации модели и её упрощения. Необходимо обосновать возможность описания и целесообразность учета выбранных на предыдущем этапе физико-химических воздействий на исследуемый объект. Если в динамических моделях большее значение имеет учет физических воздействий, то в статических --- химических воздействий;
    \item формулируется на языке математики общее уравнение, описывающее состояние объекта в заданный период времени, с учетом выделенных физико-химических воздействий;
    \item конкретизируется описание состояния объекта, однозначо выделяются входные, выходные, промежуточные переменные, а также параметры модели; 
    \item при формализации модели должно выполняться условие единственности решения, в силу чего ряд входных данных может быть доопределен, либо исключен.
\end{enumerate} 

Если динамическая модель сложна, то часто выделяют состояние системы, вносящее значительный вклад в решение задачи. В этом случае динамическую систему моделируют с помощью конечного числа статических систем.

Динамическое моделирование ресурсозатратно, но дает более точное представление о развитии процесса реального мира. Статическое моделирование описывает процесс дискретно, то есть не дает полной картины его развития. При этом демострирует качественное (а не количественное) решение при низкой ресурсозатратности.

\newpage
\section{Классификация динамических систем}
В системе могут изменяться два и более объектов, кроме того возможно взаимодействие (обмен данными) между объектами, что говорит о сложных структуре и поведении динамических систем. 

Различают простые динамические системы, структурно-сложные динамические системы, сложные динамические системы, меняющие поведение во времени, структурно-сложные гибридные системы.

\subsection{Простые динамические системы}
Определение: обычно понимается система, поведение которой задается совопкупностью обыкновенных дифференциальных уравнений в форме Коши с достаточно гладкими правыми частями, обеспечивающими существование и единственность решения. Примерами объектов являются: современная баллистическая задача, задача с бассейном (в одну трубу вливается, в другую вылевается) и др.

В ряде случаев система может быть представлена в виде нескольких простых динамических систем. Тогда такая система тоже считается просто динамической. 

\subsection{Структурно-сложные динамические системы}
Большинство технических и природных систем являются более сложными, при наличии нескольких взаимодействующих компонент системы говорят о структрурно-сложной динамической системе.

Уже отмечалось, что структура современных моделей соответствует структуре изучаемого объекта, но функционирование каждой компоненты системы скрыто от наблюдателя. Известны только входные (в ряде случаев выходные) данные и структура системы. Модель представляется в виде черного ящика с указанием количества компонентов и схемы их взаимодействия:

\begin{center}
    картинка (черный ящик с четыремя компонентами; $\bar{X}$, $\bar{Y}$)
\end{center}

При моделировании работы каждой компоненты можно получать сведения об изменении их состояния в т.ч. с учетом взаимодействий. При параллельном функционировании компонент в системе требуется синхронизация такого взаимодействия.

\subsection{Сложные динамические системы, меняющие поведение во времени}
Одной из черт сложного поведения системы является наличие у системы нескольких сменяющих друг друга состояний. Такие изменения можно представить блочной стуктурой и описывать как простую динамическую систему с определенными параметрами нагружения на разных уровнях. В случае периодической нагрузки это оправдано, но в случае стохастического изменения поведения описание оказывается сложным.

\begin{center}
    картинка с уровнями (ступеньками)
\end{center}

Непрерывное изменение поведения системы можно дискретизировать, понимая, что на границе блоков необходимо исследовать переходную функцию. В противном случае модель будет иметь значительную погрешность.\\

\emph{Погрешность моделирования} --- осознанное пренебрежение при построении матмодели. Далее обозначется как $\delta_{mod}$.\\

Причинами разрывов функций может быть использование недостаточно гладких функий для описания дискретного блока. Такие системы называются \emph{гибридными}, и помимо описания разных стилей поведения системы требует исследования функций, формализующих это поведение.

\subsection{Структурно-сложная гибридная система}
Соединяет черты структурно-сложных и гибридных систем. Сложность моделирования делает почти невозможной использование аналитических моделей в силу чего такой класс систем предполагает построение имитационных моделей.

\newpage
\subsection{}
Рассматриваемые выше статические и динамические модели описывают однокомпонентные системы и относятся к классу аналитических, учитывают физико-химические воздействия на систему. Если система много-компонента, то есть описывается взаимодействием ее составных частей (объектов), аналитические модели оказываются слишком трудоемкими и ресурсозатратными при их реализации на ЦВМ. Альтернативой является имитационное моделирование, не учитывающее конкретных физико-химических процессов, протекающих в системе, а лишь приближающее закон изменения входных данных при получении определенного набора выходных данных.

В основе идеи имитационного моделирования лежит идея \enquote{черного ящика}. То есть имитационное моделирование сводится к решению задач идентификации. Представление оператора A унифицируется.

На систему может действовать набор управляющих воздействий (вектор U), под действием которых динамически меняется набор параметров системы. Необходимость управляющих воздействий и их изменений возникает в связи с несоблюдением ограничений, наложенных как на входные так и на выходные данные. Таким образом, должны быть сформулированы ограничения для входных и выходных данных.

На рисунке представлен так называемый агрегат, соответствующий одной компоненте в системе. А компонент может быть несколько. А их функционирование как последовательным, так и параллельным или последовательно-параллельным.

Под эмуляцией понимают точное повторение функциональных возможностей системы с использованием иного алгоритма преобразования входных данных в выходные. Имитация с некоторой точностью воспроизводит функциональные возможности системы, применяя иной алгоритм преобразования входных данных в выходные (аппроксимационный метод). Симуляция  - настройка параметров системы на основе ряда входных данных с целью получения хотя бы грубого приближения.

Таким образом, симуляционным моделированием обычно называют стохастическую настройку системы с целью получения приближенного (иногда только качественного) результата. В ряде задач большего и не требуется. В случае имитационного моделирования параметры системы рассчитываются, с целью получения выходных данных с незначительной погрешностью.

\subsection{Модельное время}
При функционировании объекта реального мира в течение заданного или формируемого периода времени говорят о реальном времени. Время, затраченное на программную реализацию модели функционирующего объекта, называют вычислительным временем. Реальное время измеряется в единицах, указанных при постановке задачи. Вычислительное время, как правило, на несколько порядков меньше и измеряется в секундах или долях секунды. Модельное время измеряется в единицах реального времени. При этом оно дискретно: каждый квант модельного времени соответствует переходу системы из одного состояния в другое.\\

\emph{Модельное время} --- это дискретное время, измеряемое в единицах реального времени, связанное с вычислительным временем и используемое для внешней и внутренней синхронизации модели.\\

Очевидна невозможность последовательного моделирования компонент системы, а параллельная имитация не соответствует структуре задачи, в силу чего принято использовать следующие схемы организации квазипараллелизма (псевдопараллелизма):

\begin{enumerate}
	\item активностями;
	\item событиями;
	\item транзактами;
	\item процессами;
	\item агрегатами.
\end{enumerate}

В этом случае модельное время оказывается связанным с вычислительным временем, что способствует внутренней и внешней синхронизации модели. При этом делает возможным моделирование взаимодействия компонент.

\newpage
\section{Преобразование формального описания в имитационную модель}
Функции, как правило, возложены на управляющую программу моделирования (УПМ).

\subsection{Внутренняя синхронизация компонент модели}
Переход от формального описания к имитационной модели осуществляется на основе декомпозиции сложной системы на составные части. Для каждой компоненты кроме её формального описания устанавливают временную координату. Таким образом, УПМ реализует операторы коррекции модельного времени.

Если требуется дополнительная декомпозиция компонент модели (активностей, событий, процессов, агрегатов) и шага модельного времени, то может возникнуть необходимость в декомпозиции формального описания алгоритма.

Для транзактного способа имитации процедура декомпозиции не требуется, так как просто меняется количество заявок и интенсивность их поступления. 

Агрегатный подход по организации внутренней синхронизации похож на транзактный, с той разницей, что вместо задержки, имитирующей выполнение программы, реализуется тот или иной вычислительный алогоритм.

При моделировании событиями любое дополнительное событие включается в расписание в виде времени возникновения события.

При моделировании процессами и активностями внутренняя синхронизация обеспечивается с помощью \enquote{семафоров} --- глобальных переменных, значения которых либо разрешают, либо запрещают запуск программы-имитатора. В случае запрета активизации все компоненты модели помещатся в очередь. Далее исполняются по очереди в соответствии с заданными приоритетами.

Для транзактного способа имитации средством внутренней синхронизации компонент являются сами транзактные очереди.

При агрегатном способе дополнительно огранизуются (программно) управляющие сигналы.

Остальные подходы требуют не только активирующих сигналов, но и, в ряде случаев, контроля за ходом имитации, проверок окончания счета, контроля за взаимодействием компонентов.

\subsection{Внешняя синхронизация компонент модели}
Используются операторы синхронизации компонент в системе дискретных событий. При детальном описании компонент реализуется оператор внезапного перехода в ждущий режим или оператор перевода в ждущий режим до выполнения условия. Такими операторами функционирование модели делится на части, в рамках которых УПМ может запустить иной алгоритм описания компонент.

Для транзактного и агрегатного способа внешняя синхронизация не требуется в силу унификации решаемых задач.

\subsection{Синхронизация моделей управления информацией}
Важно, чтобы результаты функционирования одной компоненты или процесса появились не раньше времени $\tau_{ij}$, и были использованы другой компонентой не позже времени $\tau_{ij} + \Delta \tau_{ij}$. В связи с этим используют глобальную переменную, которая активирует процесс записи-считывания данных в информационном поле.

При имитации активностями и событиями допольнительная процедура не требуется. В остальных случаях необходимо осуществлять обращение к УПМ.

\subsection{Решение конфликтных ситуаций в модели}
Аппарат внешней и внутренней синхронизации, а также синхронизация моделей управления информацией с помощью операторов останова и запуска системы разрешает конфликты на основе системы приоритетов.

Особое место занимают системы, в которых компоненты динамически меняют своё поведение. Как правило составляются иерархические структуры, позволяющие УПМ найти оптимальный маршрут, разрешив при этом конфликтную ситуацию. В этом случае дополнительно требуются операторы, инициирующие активности, события, процессы, а также деинициализирущие их.

\subsection{Организация контроля за ходом имитации}
Помимо определенных задачей остановок или запусков системы возможны энергосбои и пр. Необходимо провести журнализацию состояния модели, то есть сохранить значения, определяющие модель, состояние оперативной памяти и т.д. При восстановлении условий моделирования работа будет продолжена без потерь (включая состояние УПМ).

При транзактном способе оперативный контроль ??? блоком сбора статистики. При агрегатном --- по аналогии с транзактным. Прочие методы огранизуются просто, и могут быть созданы программистом в рамке работы над УПМ.

\subsection{Огранизация сбора статистики}
Устанавливаются так называемые \enquote{наблюдатели-модели}: используются все моменты времени, выбранные установленной моделью для принудительного окончания счета, для обмена данными. Это функция сбора статистики, например:
\begin{enumerate}
    \item число взаимодействий компонент модели в некоторый период времени;
    \item среднее время пребывания заявки в системе;
    \item количество потерянных заявок;
    \item \ldots
\end{enumerate}
Большинство функций реализовано аналитически.\\

Принято создавать наблюдателя как единицу модели, например наблюдатель-активность или наблюдатель-процесс. Как правило, функция автоматически реализована в системе, но может быть расширена программистом.

\subsection{Окончание имитации}
Аналогично реализуется блок окончания моделирования. Он может отражать последовательность действий по контролю за моментам окончания имитации и начала обработки результатов моделирования. Кроме того, реализует планирование очередного имитационного эксперимента.

\subsection{Документирование}
Протоколирует результат моделирования: описание имитационной модели, входных, промежуточных и прочих данных.


\newpage
\section{Транзактный способ организации псевдопараллелизма}
Общем особенностью всех транзактных трансляторов являются понятия: \emph{транзакция} (заявка), \emph{устройство}, \emph{канал} (по которому заявки поступают в систему) и \emph{очередь}, а также некоторое количество единиц памяти в общем накопителе.

\subsection{Оператор генерации транзактов GPSS}
\texttt{GENERATE 12, 4, 50, 5, 1} --- будут сгенерированы случайные величины по равномерному закону распределения в диапозоне: $[]$. Первый транзакт появится с задержкой 50 единиц модельного времени. Всего будет создано 5 транзактов. Приоритет транзактов равен единице. Если 5 не написать (но запятую оставить), то количество транзактов не ограничено.\\

\texttt{GENERATE 12, FN\$FFF, 50, 5, 1} --- интеравал времени между транзактами есть целая часть числа 12 умножить на \texttt{\$FFF}.\\

\texttt{FNK FUNCTION, R, N1, C4 0.0/0.1, 0.8/0.5, 1.6/1, 0, 1.9} --- описание функции, аргуметом которой является случайная величина, равномерно распределенная в интрвале $[0, 1]$. Функция задана таблично четыремя точками.\\

\texttt{SEIZE PLOT} --- занятие устройства \texttt{PLOT} транзактом.\\

\texttt{RELEASE PLOT} --- освобождение устройства \texttt{PLOT} уже обслуженным транзактом.
\\

\texttt{ADVANCE A, B} --- задержка транзакта на время, длительность которого определяют парамеры: $t \in [A - B, A + B]$.\\

\texttt{QUEUE SQV}--- оператор огранизации очереди (\texttt{SQV} --- это имя). При появлении каждого транзакта длина очереди увеличивается на единицу.\\

\texttt{DEPART SQV} --- освободить одно место в очереди.\\

\texttt{ENTER MEM, 12} --- занятие транзактом двенадцати единиц памяти в накомителе (\texttt{MEM} --- это тоже название).\\

\texttt{LEAVE MEM, *2} --- освобождает несколько единиц памяти в накопителе (количество единиц хранится во втором параметре транзакта).\\

\texttt{TRANSFER, MET} --- безусловный переход по метке \texttt{MET}.\\

\texttt{TRANSFER BOTH, L1, L2} --- переход по метке \texttt{L1} если это возможно, иначе в \texttt{L2}. Если и это невозможно, то транзакт задерживается до следующего момента дискретного модельного  времени.

\newpage
\section{Имитационное моделирование}
В отличие от процесса моделирования статических или простых динамических систем, в имитационном моделировании участвует несколько специалистов (а, в ряде случаев, структурных подразделений). Различают квалификации специалистов предметной области, математиков и IT-специалистов.\\

Пунткы методики:
\begin{enumerate}
    \item постановка задачи в терминах предметной области, классификация сложной динамической системы (структурно-сложная, меняющая поведение во времени, гибридная и др.);
    \item построение структурной схемы исследуемой системы с целью выделения компонент и определения взаимодействия между ними; 
    \item по схеме уточняются известные входные данные, параметры системы, определяются неизвестные промежуточные переменные, выходные данные, определяются ограничения, обратная (положительная или отрицательная) связь, управляющие воздействия на систему, начальные и краевые условия задачи. На этом этапе модель окончательно классифицируется как сложная, предварительно выбирается схема организации квазипараллелизма модели;
    \item вместе со специалистом в предметной области строится концептуальная модель исследуемой системы;
    \item преобразование концептуальной модели в имитационную;
    \item уточнение значений входных переменных и параметров системы или получение методики оценивания этих значений;
    \item формализация модели;
    \item отладка модели на ЦВМ;
    \item счет по программе;
    \item проверка качества модели на практике, оценка необходимости модификации модели;
    \item совместно со специалистами предметной области и IT-специалистами выявляются \enquote{узкие места} модели.
    
    Для сокращения обсуждения уточнение модели проводится начиная с этапа концептуального проектирования. В редких случаях дефект может быть уже на этапе описания предметной области (первый пункт). 

    \item проведени вычислительного эксперимента с использованием построенной математической модели.
\end{enumerate}

\subsection{Преобразование концептуальной модели в имитационную}
На входе имеется предварительная схема, построенная на основе описания предметной области. На этом этапе уточняется набор входных и выходных данных и параметров системы, методики их определения, ограничения, обратная связь (при наличии) и воздействия окружающей среды или управляющия воздействия.

На первом этапе задача была поставлена в терминах предметной области без учета возможностей вычислительной техники и современного математического аппарата. Учитывая современные возможности реализации, ряд компонент могут быть объединены либо, напротив, декомпозированы. Таким образом, структура модели может значительно отличаться от структуры исследуемой системы, что, как правило, упрощает решение задачи, снижая его точность.

Система представляется в виде набора \enquote{черных ящиков}, для каждого из которых \textbf{однозначно} определен набор входных и выходных данных и алгоритм преобрзования входных данных в выходные.

На основе уточненной схемы выбирается (уточняется) схехма организации псевдо-параллелизма, соответствующая реализуемости модели.

Итогом этапа является выбор схемы организации псевдо-параллелизма, и для \emph{качественных решений}: событийный и транзактный подход, а для получения адекватных \emph{количественных оценок}: моделирование активностями и агрегатами, с целью получения \emph{прогноза} --- как правило, процессный подход.

\newpage
\section{Основные понятия систем массового обслуживания}
В общем случае постановка задачи теории СМО записывается так: в систему поступают заявки, сгенерированные по одному или разным законам распределения (известны заранее). Заявки обрабатываются втечение случайного времени (как правило, определяется интенсивностью обработки). Для обработки заявок выделяются пять секунд $n$ устройств (или \enquote{каналов}). Необходимо по таким параметрам:
\begin{enumerate}
    \item решить прямую задачу
    \begin{enumerate}
        \item среднее время нахождения заявки в очереди;
        \item скорость обработки заявок в системе;
        \item число потерянных заявок;
    \end{enumerate}
    \item решить обратую задачу
    \begin{enumerate}
        \item определить число устройств, необходимое для обработки заявок с заданной вероятностью (обычно $\in [0.9, 1.0]$).\\
    \end{enumerate}
\end{enumerate}

Различают системы с ограниченным размером очередии без ограничений. Принципиально случай ограничения очереди не отличается от неограниченной очереди за исключением учета потерянных заявок. Если для модели этот показатель принципиален, то организуют очередь с ограничением.

\subsection{Задача анализа СМО}
Анализ сводится к оцениванию показателей эффективной работы системы на основе которых могут быть приняты управленческие решения, обеспечивающие эффективную работу системы (н: увеличить количество столиков в кафе). Такие показатели делятся на четыре группы:
\begin{enumerate}
    \item показатели, характеризиующие систему в целом: число обслуженных заявок, число занятых каналов, среднее время пребывания заявки в системе, количество заявок, ожидающих обслуживания, потерянных заявок;
    \item вторичные вероятностные показатели, которые оцениваются на основе показателе первой группы: вероятность того, что заявка будет обслужена или потеряна, вероятность пустой очереди и прочее;  
    \item экономические показатели, характеризующие стоимость потерь, связанных с необслуженными заявками и наоборот, с дополнительным обслуживанием заявок и прочее;
    \item прочие показатели, затраты ресурсов на обработку (потерю) заявок.
\end{enumerate}
Показатели 1-3 унифицированы, включены в статистику инструментальных средств анализа СМО. Показатели четвертой группы могут быть рассчитаны программно.

\subsection{Математическое описание СМО}
СМО рассматриваются как некоторые физические системы с дискретными состояниями $S_0, S_1, \ldots, S_n$, функционирующие при том в реальном времени $t$. Число таких состояний может быть как конечно, так и бесконечно. Для анализа системы строится граф переходов, где вершинами являются состояния, а ребрам соответствуют вероятности переходов из одного состояния в другое. Переход может быть возвратным и невозвратным (одно- или двунаправленное ребро). По аналогии с графом составляется таблица переходов (матрица смежности).

Переход из состояния в состояние происходит в случайный момент времени $t_{ij}$. В ряде случаев переходы описывают в виде потоков: если заявки однотипны, то выделяют поток установления на обслуживание, поток прибывания в очереди, поток обслуживания заявок и поток освобождения очереди. То есть, математическое моделирование СМО сводится к описанию на основе Марковских Цепей (не принципиально каким образом заявка оказалась в состоянии $S_i$, но принципиально в какое следующее состояние и с какой вероятностью она перейдет).\\

ПРОПУЩЕНА ЛЕКЦИЯ\\

Рассмотрим СМО в некоторый момент времени $t$ и, задав некоторый малый промежуток времени $\Delta t$, оценим вероятность того, что в момент времени $t + \Delta t$ система не выйдет из состояния $S_0$. Возможно два случая (???)

\begin{enumerate}
    \item Пусть в момент времени $t$ с некоторой вероятностью $p_0(t)$ система находилась в состоянии $S_0$. Возможно суммировать переход из $S_0$ в $S_1$ и из $S_1$ в $S_2$ или сразу перейти в состояние $S_2$. Вероятность того, что мы останемся в состоянии $S_0$ равна $p_{00} = 1 - (p_{01} + p_{02})$, где $p_{ij}$ --- вероятность перехода из состояния $S_i$ в состояние $S_j$.:
    
    Индуктивно, $\lambda_0 = 1 - (\lambda_{01} + \lambda_{02})$ --- интенсивность переходов, $p = \Delta t \lambda$. Итого, имеем:

    \[ \Delta t \lambda_0 = \Delta t \left[1 - (\lambda_{01} + \lambda_{02}) \right], \]
    \[ \hat{p_{00}} = 1 - \Delta t (\lambda_{01} + \lambda_{02}). \]

    \item Система с вероятностью $p_1(t)$ находится в состоянии $S_1$ или в состоянии $S_2$ с вероятностью $p_2(t)$. Необходимо оценить вероятность того, что за время $\Delta t$ система вернется в состояние $S_0$.
    
    \[ p_0 (t + \Delta t) = p_1 (t) \cdot \lambda_{10} \Delta t + p_2 (t) \cdot \lambda_{20} \Delta t + p_0 (t) \cdot [ 1 - (\lambda_{01} + \lambda_{02}) \Delta t],\]
    \[ \frac{p_0 (t + \Delta t)}{\Delta t} = - p_0 (t) + p_1(t) \lambda_{10} + p_2(t) \lambda_{20} - (\lambda_{01} + \lambda_{02}) p_0(t), \]
    \[ p'_0(t) \approx p_1 (t) \lambda_{10} + p_2 (t) \lambda_{20} - (\lambda_{01} + \lambda_{02}) p_0 (t),\]
    \[ p_1 (t) \lambda_{10} + p_2 (t) \lambda_{20} - (\lambda_{01} + \lambda_{02}) p_0 (t) \underset{\Delta t \to 0}{\to}  p'_0(t). \]
\end{enumerate}

Индуктивно расспространяя рассуждения на другие состояния системы, получим:
\[
\begin{cases}
    p'_0(t) = p_1 (t) \lambda_{10} + p_2 (t) \lambda_{20} - (\lambda_{01} + \lambda_{02}) p_0 (t),\\
    p'_1(t) = p_0 (t) \lambda_{01} + p_3 (t) \lambda_{31} - (\lambda_{10} + \lambda_{13}) p_1 (t),\\
    p'_2(t) = p_0 (t) \lambda_{02} + p_3 (t) \lambda_{32} - (\lambda_{20} + \lambda_{23}) p_2 (t),\\
    p'_3(t) = p_1 (t) \lambda_{13} + p_2 (t) \lambda_{23} - (\lambda_{31} + \lambda_{32}) p_3 (t).\\
\end{cases}
\]

Из этой системы следует, что вероятности $p_i(t) \to p_i$, $i = \overline{0, \ldots, n}$. Причем эти вероятности с течением времени стремятся к некоторомым значениям $p_i$, называемым \emph{предельной (финальной) вероятностью}. В ряде случаев требуется оценить именно финальные вероятности. Т.к. финальные вероятности константы, то

\[
\begin{cases} 
    \lambda_{10} p_1(t^*) + \lambda_{20} p_2(t^*) = (\lambda_{01} + \lambda_{02}) p_0 (t^*),\\
    \lambda_{01} p_0(t^*) + \lambda_{31} p_3(t^*) = (\lambda_{01} + \lambda_{13}) p_1(t^*),\\
    \lambda_{02} p_0(t*) + \lambda_{32} p_3(t^*) = (\lambda_{20} + \lambda_{23}) p_2(t^*),\\
    \lambda_{13} p_1 (t^*) + \lambda_{23} p_2(t^*) = (\lambda_{31} + \lambda_{32}) p_3(t^*)).
\end{cases}
\]

При условии известных интенсивностей перехода и системы выше оцениваются величины предельных вероятностей.

\begin{framed}
    \textbf{Важно:} выкладки выше приведены для конкретной системы (четыре состояния $S_i$), а не в общей форме!
\end{framed}
%\smallskip

$\triangleright$ Пример. Четыре состояния $S_0, S_1, S_2, S_3$. Вероятности переходов: 

\begin{align*}
    &\lambda_{01} = 1, ~~~\lambda_{10} = 2,\\
    &\lambda_{02} = 2, ~~~\lambda_{20} = 3,\\
    &\lambda_{13} = 2, ~~~\lambda_{31} = 3,\\
    &\lambda_{23} = 2, ~~~\lambda_{32} = 3.
\end{align*}

В ряде случаев уравнения в системе могут оказаться линейно-зависимыми. В таких случаях систему определяют краевыми или начальными условиями:

\[
\begin{cases}
    3 p_0 = 2p_1 + 3p_2,\\
    4p_1 = p_0 + 3p_3,\\
    4p_2 = 2p_0 + 2p_3,\\
    \underline{p_1 + p_2 + p_3 = 1}.
\end{cases}
\implies
\begin{cases}
    \hat{p_0} = 0.10,\\
    \hat{p_1} = 0.20,\\
    \hat{p_2} = 0.27,\\
    \hat{p_3} = 0.13.
\end{cases}
\]\\

\enquote{Найти средний чистый доход от эксплуатации в стационарном режиме исследуемой системы, если исправная работа первого и второго узла приносит доход 10 и 6 у/е соотвественно, а ремонт требует затрат 4 и 2 у/e соответственно. Оценить имеющуюся возможность уменьшения вдвое среднего времени ремонта каждого из двух узлов, если при этом вдвое увеличиваются затраты на ремонт каждого узла в единицу времени}.\\

Решим задачу с использованием постренной системы. Примем обозначения: $S_0$ --- база, $S_1$ --- первый станок, $S_2$ --- второй станок, $S_3$ --- ремонт.
\begin{align*}
    &p_0 + p_3 = 0.4 + 0.2 = 0.6,   &\text{(I)}\\
    &p_0 + p_2 = 0.4 + 0.27 = 0.67, &\text{(II)}\\
    &p_1 + p_3 = 0.2 + 0.13 = 0.33, &\text{(I, ремонт)}\\
    &p_2 + p_3 = 0.27 + 0.13 = 0.4. &\text{(II, ремонт)}
\end{align*}
Итого, доход равен $0.6 \cdot 10 + 0.67 \cdot 6 - 0.33 \cdot 4 - 0.4 \cdot 2 = 7.9$ у/е. $\triangleleft$

\newpage
\section{Стохастическое моделирование}
\subsection{Основные задачи математической статистики и теории вероятности. Вероятностные и статистические модели}
Мат. статистика (МС) и теория вероятностей (ТВ) изучает математические модели случайных явлений или процессов. Задачи МС являются обратными по отношению к задачам ТВ: в ТВ после задания некоторого случайного явления требуется рассчитать вероятностные характеристики этого явления или процесса. Приведенные выше примеры являются вероятностными моделями. В задачах МС известны результаты эксперимента, и требуется оценить неизвестные параметры модели.

Вторая задача --- это задача \emph{интервального} оценивания неизвестных параметров. При этом требуется построить интервал с неизвестными границами, в который анализируемый параметр попадет с заданной вероятностью.  

\begin{framed}
    \textbf{Напоминание:} доверительный интервал и коэффициент доверия:
    \[ P \left\{ \underline{a} \leq a \leq \overline{a} \right\} = \gamma, \]
    $\underline{a}, \overline{a}$ --- нижняя и верхняя границы доверительного интервала;\\
    $\gamma$ --- коэффицинет доверия. Обычно выбирается равным одному из значений: 0.9, 0.95, 0.99, 0.995, 0.999, 0.9995.
\end{framed}

Третья задача МС --- задача проверки статистических гипотез, в которой требуется на основе данных эксперимента проверить то или иное предположение.

Следует отличать статистические гипотезы от бытовых (н: \enquote{После обеда пойдет дождь}). Статистическая гипотеза является формализацией бытовой, и, формулируется относительно закона распределения параметров модели или, непосредственно, самих параметров модели.

В МС часто используется выборочная терминология, основанная не следующей \enquote{урновой} схеме: пусть имеется урна, содержащая $N$ чисел $x_1, x_2, \ldots, x_N$, скрытых от наблюдателя. Весь этот набор называется \emph{генеральной совокупностью}. Из генеральной совокупности случайным образом выбирается конечное число $n$ значений. Такой набор значений $x_1, \ldots, x_n$ называется выборкой объема $n$ из генеральной совокупности. Различают выборки с и без \emph{возвращения} (обратно в генеральную совокупность). Если выборка производится с возвращением, то случайные величины независимы. В противном случае зависимы. В первом случае такой набор называется \emph{независимой повторной случайной выборкой} объема $n$.

В случае бесконечной генеральной совокупности терминология сохраняется, но в этом случае разница между выборками с возвращением и без фактически изсчезает.\\

Если упорядочить выборку $x_1, \ldots, x_n$ в порядке неубывания (реже: невозрастания):

\[ x^{(1)}, x^{(2)}, \ldots, x^{(n)}, \]
\[ x^{(1)} \leq x^{(2)} \leq \ldots \leq x^{(n)}, \]
то такая выборка назыается \emph{вариационным рядом}.

\begin{framed}
    \textbf{Важно:} не путать с временным рядом!
\end{framed}

Для описания выборки в целом и вариационного ряда в частности используют империческую функцию распределения.\\

Прежде чем дать общее описание, рассмотрим $\triangleright$ пример. Рассмотрим выборку  $\{ 1025, 1031, 1048, 1092, 1216, 1235, 1257, 1274, 1299. 1312 \}$.
\[ \hat{F}_k(x) = 
\begin{cases}
    0, ~~~x \leq x_0,\\
    \frac{j}{n}, ~~~x_0 < x < x_{max},\\
    1, ~~~ x \geq x_{max},
\end{cases}
\]
где $n$ --- общее число точек в выборке, $k$ --- число разбиений, $j$ --- число значений, попавших в диапазон $[x_0, x_i]$ ($i$ --- порядковый номер интервала). Можно приблизить имперической функцией
\[ \hat{F}_k (x) = \frac{r(x)}{n}, \]
$r(x)$ --- число возможных чисел в выборке, для которых $x \in [x_0, x_i]$. 

По теореме Гливено-Кантелли при $n \to \infty$ с вероятностью 1 выполняется соотношение

\[ \sup_x \left| \hat{F}_n(x) - G(x) \right| \to 0. \]

На основании теоремы Гливено-Кантелли можно сделать вывод, что эмперическая функция распределения сходится к теоретической.
$\triangleleft$

%\begin{center}
%тут картинка
%\end{center}
  
\end{document}

