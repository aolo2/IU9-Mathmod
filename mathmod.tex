\documentclass[12pt]{article}

\RequirePackage[l2tabu, orthodox]{nag}

\usepackage[utf8]{inputenc}
\usepackage[russian]{babel}
\usepackage{blindtext, csquotes, indentfirst}

\title{Лекции по математическому моделированию}
\author{А.Б. Домрачева}
\date{Весенний семестр 2019 г.}

\begin{document}
\maketitle
\thispagestyle{empty}

\newpage
\tableofcontents
\thispagestyle{empty}

\newpage
% . Математическое моделирование и решение инженерных задач с применением ЭВМ
\section{Лекция 1}
\subsection{Характеристики инженерных задач}
Прикладные задачи классифицируют как экономические, инженереные и научные. Отличительные особенности инженерных задач:

\begin{enumerate}
    \item инженерные задачи имеют ярко-выраженную практическую направленность (создание новых конструкций, разработка техпроцесса, минимизация затрат на производство изделий и т.д.);
    \item для инженерных задач характерна необходимость доведения раеультатов до конкретных чисел, на основании которых можно принимать решения;
    \item для инженерных задач характерен значительный объем выполняемой вычислительной работы, при этом набор методов и их реализаций конечен;
    \item для инженерных задач характерна неохдимость создания достаточно сложных математических моделей с использованием современных вычислительных методов. Экономические задачи используют простые типовые модели. Научные задачи, наоборот, выходят за рамки сложных, но типовых можелей и требуют разработки или модификации математического аппарата для их решения и реализации;
    \item инженерные задачи решаются, как правило, специалистами в предметном области (не программистами, не мат-моделистами, не прикладными математиками).
\end{enumerate}

\begin{quotation}
    \enquote{Нельзя научить делать открытия, но можно подготовить открытие}
    \rightline{{--- А.Б. Домрачева}}
\end{quotation}

За это отвечает теория инженерного эксперимента --- один из разделов математического моделирования. В свою очередь, теория инженерного эксперимента делится на теории планирования эксперимента и проведения эксперимента и обработку результатов эксперимента.

\subsection{Проблемы, возникающие при решении инженерных задач}
Проблема первая. Характеристики объектов испытаний, которые требуется определить в разультате эксперимента, часто оказываютяс недоступными непосредственному измерению.

Проблема вторая. Процессы функционирования объекта исследования часто имеют сложный динамический характер и подвержены существенным влияениям изменяющихся условий внешней среды.

Проблема третья. При испытании сложных комплексом необходимо учитывать влияние испытательного, регистрирующего, управляющего оборудования (погрешность измерения).\\

Таким образом, эксперимент может оказать продолжительным и дорогостоящим. В связи с этим необходим системный подход, предполагающий решение описанных выше проблем с обеспечением необходимой точности решения.\\

\textit{Математическое моделирование} --- метод исследования объектов и процессов реального мира с помощью их приближенных описаний под средством формул (равенств, неравенств, уравнений, логических структур). Такое описание называют математическими моделями.\\

Математическая модель --- неотъемлемая часть эксперимента, исследуемая и уже известная техниками (?)  экономической характеристики объекта.
Выделяют класс плохо-формализуемых задач, для кототрого известны собственные методы построения моделей.

Процесс создания математической модели:
\begin{enumerate}
    \item построение математической модели;
    \item постановка исследования и решение вычислительных задач, реализующих модель;
    \item проверка качества моделей на практике, модификация моделей.
\end{enumerate}

\subsection{Построение математической модели ч. 1}
Имея результаты эксперимента, можно установить связь между этими величинами, описываемую на языке математики. При выборе известной математическоймодели (модель должа быть достаточно полной для изучения свойств исследования или объекта и достаточно простой для реализации на ЭВМ) выбор учитываемых существернных и несущественных факторов носит исследовательский характер: как правило подтвердается характеристиками. В противаном случае компромисс матмодели будет определен неверно.

Прежде чем использовать ту или иную известную матмодель необходимо определить тип модели: аналитическая, гипотетическая, имитационная, статическая или динамическая.

\begin{itemize}
    \item[]Гипотетическая модель --- еще не подтвержденный процесс.
    \item[]Статическая модель --- описывает явления в предположении, что процесс завершен.
    \item[]Динамическая модель --- как протекают явления или процессы раельного мира при переходе от одного состояния к другому.
    \item[]Имитационная модель --- позволяет имитировать поведение реальной системы на ЦВМ в заданный или формируемый период времени с определенным набором входных данных.
    \item[]Аналитическая модель --- формализация явления или процесса реального мира, учитывающая физико-химической воздействия, протекающие в система.
\end{itemize}

\textbf{Пример: баллистическая задача.} Пусть тело (шар) брошено под ушлом $\alpha$ к поверхности Земли со скоростью $v_0$. Пренебрегая сопротивлением воздуха, считая Землю плоской, а $g = const$, определить дальность полета.\\

Модель Галилея.

\begin{center}
    картинка
\end{center}

\[\begin{cases}
    x = v_0 \cdot \cos \alpha \cdot t,\\
    y = v_0 \cdot \sin \alpha \cdot t - g \frac{t^2}{2}.    
\end{cases}\]

\[\begin{cases}
    x_1 = 0,\\
    y_1 = 0
\end{cases},
\begin{cases}
    x_2 = \tg \alpha \frac{2 v_o \cos^2 \alpha}{g},\\
    y_2 = \frac{-g}{2 v_0^2 \cos^2 \alpha} x^2 + (\tg \alpha) x
\end{cases}
\]\\

Модель Ньютона: постановка задачи совпадает, но учитывается сила лобового сопротивления воздуха $F = -\beta v^2$, где $\beta = \frac{c \rho S}{2}$ --- коэффициент сопротивления воздуха.

\begin{center}
    картинка
\end{center}

\[
\begin{cases}
    m \frac{du}{dt} = - \beta u \sqrt{u^2 + v^2},\\
    m \frac{dw}{dt} = - \beta w \sqrt{u^2 + v^2} - mg.
\end{cases}
\]

Модель Ньютона более точная в сравнении с моделью Галилея, но не удовлетворяет решению современной баллистической задачи (линейные и угловые параметры, всего шесть неизвестных).

\newpage
\section{Лекция 2}
\subsection{Построение математической модели ч. 2}
Имея результаты эксперимента (выраженные в числовых щзначениях) необходимо установить связь между этими величинами, описываемую формулам. Модель будут определять величины, которые можно назвать:
\begin{enumerate}
    \item[а)] исходными данными;
    \item[б)] параметрами модели;
    \item[в)] выходными данными.
\end{enumerate}

В моделировании, как правило, модель представляется в виде \enquote{черного ящика}, входные данные --- вектор $\bar{X}$, выходные данные --- вектор $\bar{Y}$, параметры --- $A$.

\begin{center}
    картинка
\end{center}

В соответствии со схемой выделяют типы решаемых задач:

\begin{itemize}
    \item по входным данным и набору параметров при известном наборе преобразований входных данных в выходные требуется найти решение;
    \item по значениям выходных данных $\bar{Y}$ и фиксированных значенияъ параметров $A$ (при извесном наборе преобразований входных данных в выходные) необходимо оценить набор оисходных данных $\bar{X}$;
\end{itemize}

В модели Галилея решение обратной задачи требует минимум двух переменных в векторе $\bar{Y}$.\\

Ранее как обратные классифицировались и задачи идентификации в узком смысле, то есть такие задачи, в которых известны наборы $\bar{X}$ и $\bar{Y}$, а также тип связи; достаточно было оценить параметры.

Задача идентификакции в широком смысле --- когда нет никаикх сведений о типе преобразования (ранее считались не имеющими решения). Пример: в СЛАУ $A\bar{x} = \bar{b}$ найти $A$ при известных $\bar{x}, \bar{b}$.\\

% Как наука моделирование зародилась в 1846 г., когда французкий астроном обнаружил Нептун на основании заонов небесной механики с использованием данных об аномалиях в движении планеты Уран.\\

Даже для простых моделей реализация используем два и более вычислительных алгоритма, то есть после формализации модели на первом этапе необъходимо проверить корректность постановки каждой задачи по Адамару-Петровскому и выбрать вычислительный метод решения каждой задачи, обсеспечив хорошую обусловленность как самой задачи, так и алгоритма.

После первого расчета по выбранной модели выясняют пригодность для описания процесса или объекта реального мира.\

Теоретические вызовы (?) и результаты, полученные из математической модели, сопоставляются с экспериментами. Может выясниться, что модель адекватна, то есть вполне точно описывает объект исследования, иначе требующтся модификации. На этом этапе решение принимают специалисты в предметной области, оценивая полученную точноть и рентабельность модфикаций модели.\\

Для проведения эксперимента:
\begin{itemize}
    \item составляется план;
    \item создается экспериментальная установка;
    \item выполняются контрольные эксперименты;
    \item проводятся серийные опыты;
    \item обрабатываеются полученные экспериментальные данные и их результаты.
\end{itemize}

Натурные эксперименты как правило дороги, часто продолжительны. Дешевой и быстродейственной альтренативой натруному эксперименту явлется \emph{вычислительный эксперимент}, в основе которого елжит реализация математическом модели на ЦВМ.

\texttt{TODO} коллизия!

Компромиссом является так называемый полунатурный эксперимент, где в качестве модели используются физическая модель объекта, доступные характеристики оцениваются экспериментально (н: число Маха в аэродинамической трубе), другие - програмно.

Преимущества вычислительного эксперимента:
\begin{itemize}
    \item[+] дешевле
    \item[+] безопасен
    \item[+] можно повторить
    \item[+] предполагает моделирование условий, которые нельзя создать в лаборатории
\end{itemize}

Ограничения:
\begin{itemize}
    \item[-] применимость разультата ограничена рамками используемой математической модели
\end{itemize}


С (???), а также оптимизацию по заданному критурию или ряду параметров. Для получения оптимального результата может понадобиться большое количество вариантов оптимизации, что делает заведомо продожлительными и вычислительный эксперимент. Для сокращения времени эксперимента строят план вычислительного эксперимента, для чего в настоящее время существует ПО, позволяющее автоматизировать процесс планирования и проведения эксперимента (т.н. проблемно-ориентированное ПО), что вместе с вычислительной техникой и обсл (?) персоналом составляет проблемно-ориентрированную ориенционную (?) систему исследований.

\subsection{Основные этапы решения задачи с применением ЦВМ. Вычислительный эксперимент}
Решение инеженерной задачи разделяетяс на 10 этапов:
\begin{enumerate}
    \item постановка задачи;
    \item построние или модификация математической модели;
    \item постановка вычислительной задачи;
    \item предварительный предмашинный анализ свойств вычислительной задачи;
    \item выбор и построние математической модели;
    \item алгоритмизация и программирование;
    \item отладка программы;
    \item счет по программе;
    \item обработка и интерпретация результатов;
    \item использование результатов (в т.ч. для коррекции матмодели).
\end{enumerate}

\subsection{Постановка задачи}
Имеется общая прдеварительная формулировка. Установка конкретизируется с уточнением цели исследования. Определяется система координат, в которой задается система координат, входные и выходные данные. Находится компромисс между полезностью результатов и сложностью достижения цели. Задача формулизуетя на языке предметной области (но с учетом возможностей современной вычислительной техники).

\subsection{Построние математической модели}
Анализ объекта исследования. Меняется ли он динамически в заданный или формуремый период времени, носят ли переменные (входные или выходные) или алгоритм их преобразования случайный характер, либо нет. На основе анализа цели исследования оценивается необходимость решения прямой, обратной или идентификационной задачи.

\begin{enumerate}
    \item[1)] Анализируется возможность измерения известнрых согласно поставновке задачи данных; строится схема объекта исследования в виде черного ящика с учетом количества компонент и изменения поведения системы во времени.
    \item[2)] Выбираетяс тип моделируемой системы (простая, стуктурно-сложная, сложная с изменяющимся поведениям, гибридная, \ldots). На схеме указываются все измеряемые переменные и параметры, указывается способ их получепния.
    \item[3)] Определяется тип бозовой вычислительной задачи (прямая, обратная, идентификационная).
    \item[4)] Указывается классифицирующие признаки модели (статическая или динамическая, аналитическая или имитационная, вроятностная или статистическая, \ldots). 
    \item[5)] По рузальтатам классификации с учетом опыта предметной области выбирается, модифицируется или строится матмодель.
\end{enumerate}

В случае необходмимости построения матмодели учитывается опыт смежных областей, а также поиск аналогов.

При построении модели разного типа имеются принципиальные различия, в связи с чем используются типовые методики, в том чсле \emph{методика динамической модели} действий или процесса.\\

Методика динамической модели:
\begin{enumerate}
    \item[1)] Рисуется графическая схема, иллюстрирующая процесс. Например: в выбранной системе координат указываются время начала и, по возможности, время конца измерений. Альтернативой является блок-схема процесса. В схеме уточняются входные и выходные данные, выделяются переменные, которые являются выходными на одном этапе решения задачи и входными на другом.
    \item[2)] ???
    \item[3)] Определяются физико-химические воздействия на процесс \ldots ???
    \item[4)] Оценивается возможность упрощения модели. Ряд воздействий не учитывается.
    \item[5)] Формулизуется общее уравнение модели.
    \item[6)] Конкретизуется описание каждого отдельного воздействия.
\end{enumerate}

Любая модель предполагает формулировку ограничений, накалдываемых на неё. Эти ограничения указываеются на этапе постановки задачи и на графической схеме могут обозначанться как вычеркивания физико-химических воздействий на систему.

\newpage
\section{Лекция 3}
На этом этапе уточняется цель исследования и на основе цели формулизуются требования к точности вычислений результата и к точности представлений исходнных данных. В ряде случаев исходные данные также явяются результатом рассчетов (если не заданы, не могу быть измерены на основе натурного эксперимента, не выражабются через другие известные либо измеряемые характеристики аналитически). Натурный эксперимент чаще всего является дорогостоящий, объект исследования в нем --- непосредственнен. Объект исследования задачи, в ряде случаев --- продолжительный. При этом построение математической модали на основе вычислительного эксперимента приводит к основной коллизии математиечкого моделирования. В этом случае проводят полунатурный эксперимент, объектом  исследования которого является физическая модель объенкта исследования. Результат, полученный с помощью полунатурного  эксперимента может оказаться недостаточно точным. В связи с с чем появляется дополнительная вычислительная задача итерационно невытающая (?) точность оценки. В ряде случаев уточнения трубет и результат решения свей задачи, то есть в рамках моделиврования формулизуется основная вычислительная задача, и ряд вспомогательных, определяющих входные (выходные) данные и параматры модели. Проводится классификация задачи (прямая, обратная, идентификационная), в ряде случаев основная задача также разбивается на несколько вычислений, которые решить проще.

\begin{center}
    картинка
\end{center}

Каждая полученная задача анализируется на корректность постановки по Адмару-Петровскому. Если какая-либо из задач является некорректно поставленной, то ее необходимо переформулировать с целью обсеспечения существования и единственности решения, оценить обусловленность задачи, сипашунр (?) возможность обеспчения устойчивости и хорошей обусловленности решения. Можно переходить к выбору и построению численных методов и последующей алгоритмизации. Перед выбором численного метода задачи поисаются (?) решить количественно, то есть даже получить количественный результат, но с невысокой точностью. При этом становится известным диапозон значений для каждого решения и вырабатываютяс ограничения, накладываемые на данные или метод, в том числе подтверждается возможность решения задачи . Чаще рещение инеженрной задачи сводится к последовательному решению стандарных вычислительных задач для который разработаны эффективные численные методы, но возможны ситуации, когда алгоритм необходимо адаптипровать к решению задачи или создать новый метод.

Если имеется несколько методов решения, необходимо сравнить эффективноть всей совокупности методов. Для сравнения эффективности выбирают показатели по которым будут сравинвать методы и составоляется таблица эффективности: по горизонтали --- название мтодов, по вертикали --- значпения показателей этого метода.

\subsection{Алгоритмизация и программирование, отладка и счет по программе}

Численный мтеод содержить гопоно (?) принумн. (?) схему решения, в сявзи с чем на этом этапе разрабатываются вычислительный алгоритм. Алгоритм проверятся на корректность, обеспечивапется устойчивость решения (в том числе вычислительная), результат должен быть достижим за конечное число итераций.

При тестировании алгоритма осуществляется не тоолько поиск синтактических ошибок, но и провдится валидация результата на тестовых данных. Если часть данных задачи произвольна, то решается задача обратная к поставленной, результат выводится в удобной для восприятия форме многократного запуска с тестовыми данными подвивечхует (?) хорошую обусловленность решения и устойчивость. Изменениые результата оценивается с помощью допустимой погрешности. В случае оперед. (?) решения вернуться на этап анализа вычислительной задачи. При многократном запуске с измененными данными подтверждается устойчивость решения, проверяются ограничения, наложенные при постановке задачи.

\end{document}